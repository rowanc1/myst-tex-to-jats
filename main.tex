\documentclass{article}

% Title and Author Information
\title{A Simple LaTeX Article!!}
\author{Rowan Cockett\thanks{rowan@curvenote.com}}

\newcommand{\dobs}{\bm{d}_\text{obs}}
\newcommand{\dpred}{\mathbf{d}_\text{pred}\left( #1 \right)}
\newcommand{\mref}{\bm{m}_\text{ref}}

\begin{document}

\maketitle

% Abstract
\begin{abstract}
This is a simple LaTeX document showcasing the basic structure of an article. It includes a title, author information, abstract, and body. LaTeX is a typesetting system commonly used for the production of scientific and mathematical documents due to its powerful features for handling complex formatting.

You can parse and render this in MyST.
\end{abstract}

% Body of the Article
\section{Introduction}

Lorem ipsum dolor sit amet, consectetur adipiscing elit. Sed vitae justo et orci euismod fringilla.
Including latex citations \citep{smith2010}.

\section{Methodology}
Pellentesque habitant morbi tristique senectus et netus et malesuada fames ac turpis egestas.

In (\ref{maxwell}) we can see some equations for light!

The residual is the predicted data for the model, $\dpred{m}$, minus the observed data, $\dobs$. You can also calculate the predicted data for the reference model $\dpred{\mref}$.

$$
\label{maxwell}
\begin{aligned}
\nabla \times \vec{e}+\frac{\partial \vec{b}}{\partial t}&=0 \\
\nabla \times \vec{h}-\vec{j}&=\vec{s}\_{e}
\end{aligned}
$$

\section{Results}
Nulla facilisi. Duis tristique, velit et euismod cursus, augue metus feugiat velit, et tincidunt elit nulla a velit.

You can see in \autoref{fig:example} and in \autoref{tab:sample}, that cross references work.

\begin{figure}[h]
  \centering
  \includegraphics[width=0.8\linewidth]{image.png}
  \caption{Example Figure}
  \label{fig:example}
\end{figure}


\begin{table}[h]
  \centering
  \caption{Sample Table}
  \label{tab:sample}
  \begin{tabular}{|c|l|r|}
    \hline
    \textbf{ID} & \textbf{Name} & \textbf{Score} \\
    \hline
    1 & John Doe & 85 \\
    2 & Jane Smith & 92 \\
    3 & Bob Johnson & 78 \\
    \hline
  \end{tabular}
\end{table}

Some commands that are helpful are:

\begin{verbatim}
myst build main.tex --jats -o article.xml
\end{verbatim}

\section{Conclusion}
In conclusion, this example serves as a starting point for creating your LaTeX articles. Feel free to explore additional features and packages to enhance the formatting and appearance of your documents.

\bibliographystyle{plain}
\bibliography{references}

\end{document}
